\documentclass{ximera}

\title{JMM 2026: Motivation for Determinants}
\author{YOUR-NAME-HERE}

\begin{document}
\begin{abstract}
    A short activity that lets students explore the effects of a linear transformation on area to motivate determinants.
\end{abstract}
\maketitle

We've seen how linear transformations transform vectors and can be represented by matrices.  Even after looking at a handful of vectors it is often difficult to tell what the transformation actually accomplishes. This is why sometimes looking at how the transformations affect shapes can provide some valuable intuition.

\begin{center}
    \geogebra{cdzqzkgm}{950}{800}
\end{center}

How can we double the area of this rectangle?


\begin{exercise}
One way, conceptually is to double the height.  How can we do this with the matrix representation of the linear transformation?  Try changing the entries in the matrix until the image is a rectangle with the same length, but double the height.  What are the entries in the matrix?

    $$M=\begin{bmatrix}\answer{1} &\answer{0} \\\answer{0} & \answer{2}\end{bmatrix}$$

    \begin{exercise}
Another way is to make the height half as tall and the length 4 times longer. What are the entries in this matrix?

   $$M=\begin{bmatrix}\answer{4} &\answer{0} \\\answer{0} & \answer{1/2}\end{bmatrix}$$

\begin{exercise}
Now, change the $a_{12}$ entry from 0 to 1.  What happens to the image of the original rectangle?  What is the area $A$ of the transformed rectangle?  $A=\answer{4}$

\end{exercise}
\end{exercise}
\end{exercise}
\end{document}
